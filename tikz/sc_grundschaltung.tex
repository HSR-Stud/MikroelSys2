% Create the switch symbol
\makeatletter
\pgfcircdeclarebipole{}{\ctikzvalof{bipoles/interr/height 2}}{spst}{\ctikzvalof{bipoles/interr/height}}{\ctikzvalof{bipoles/interr/width}}{

    \pgfsetlinewidth{\pgfkeysvalueof{/tikz/circuitikz/bipoles/thickness}\pgfstartlinewidth}

    \pgfpathmoveto{\pgfpoint{\pgf@circ@res@left}{0pt}}
    \pgfpathlineto{\pgfpoint{.6\pgf@circ@res@right}{\pgf@circ@res@up}}
    \pgfusepath{draw}   
}
% make the shape accessible with nice syntax
\def\pgf@circ@spst@path#1{\pgf@circ@bipole@path{spst}{#1}}
\tikzset{switch/.style = {\circuitikzbasekey, /tikz/to path=\pgf@circ@spst@path, l=#1}}
\tikzset{spst/.style = {switch = #1}}
\makeatother

\begin{circuitikz}[scale=0.5]

	\draw
		(0,-2) node[sground]{} to[american voltage source] (0,-0.5)
		(0,-0.5)to (0,0)
		(0,0) 	to[short,i=$i_{in}$] (1,0)
		(1,0)	to[switch=$S_0$] (2,0)
		(2,0)	to (4,0)
		(3,0)	to[short,i=$i_c$](3,-0.5)
		(3,-0.5)to[C=$C$] (3,-2) node[sground]{}
		(4,0)	to[switch=$S_1$] (5,0)
		(5,0)	to (6,0)
		(6,0)	to[short,i=$i_{out}$] (6,-2) node[sground]{}
		
		;
		
	

\end{circuitikz}