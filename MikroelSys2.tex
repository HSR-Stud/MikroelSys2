\documentclass{scrartcl}

\include{header/zusammenfassung}
\usepackage{tikz}
\usepackage{adjustbox}
\usepackage{multicol}
\usetikzlibrary{arrows}
\usetikzlibrary{calc}
\usepackage{circuitikz}
\usepackage{tikz-timing}[2009/05/15]


% Create the switch symbol
\makeatletter
\pgfcircdeclarebipole{}{\ctikzvalof{bipoles/interr/height 2}}{spst}{\ctikzvalof{bipoles/interr/height}}{\ctikzvalof{bipoles/interr/width}}{

    \pgfsetlinewidth{\pgfkeysvalueof{/tikz/circuitikz/bipoles/thickness}\pgfstartlinewidth}

    \pgfpathmoveto{\pgfpoint{\pgf@circ@res@left}{0pt}}
    \pgfpathlineto{\pgfpoint{.6\pgf@circ@res@right}{\pgf@circ@res@up}}
    \pgfusepath{draw}   
}
% make the shape accessible with nice syntax
\def\pgf@circ@spst@path#1{\pgf@circ@bipole@path{spst}{#1}}
\tikzset{switch/.style = {\circuitikzbasekey, /tikz/to path=\pgf@circ@spst@path, l=#1}}
\tikzset{spst/.style = {switch = #1}}
\makeatother


\DeclareMathOperator{\Ref}{\text{ref}}
\DeclareMathOperator{\In}{\text{in}}
\DeclareMathOperator{\Out}{\text{out}}
\DeclareMathOperator{\Zop}{Z_{\text{op}}}
\DeclareMathOperator{\Int}{\text{int}}
\DeclareMathOperator{\Fb}{\text{fb}}
\DeclareMathOperator{\Fclk}{f_{\text{clk}}}

\title{MikroelSys2 Zusammenfassung}
\subtitle{Dozent: G.Keel}
\author{H.Badertscher, G.C.Köppel}

\begin{document}
\selectlanguage{ngerman}

\maketitle

\section{Messung von Kapazitäten}

\subsection{Stromquelle}
Prinzip: Kapazität wird mit einem Strom $I_0$ während einer fixen Zeit 
$t_0$ aufgeladen. Die Kapazität berechnet sich aus
\begin{equation*}
	V_{\Out} = C_x \cdot I_0 \cdot t_0
\end{equation*}
Diese Art der Messung benötigt eine präzise Stromquelle oder eine
Referenzkapazität. Weiter werden sehr kleine Ströme benötigt, was
diese Methode für PCB-Design unpraktikabel macht. Im IC-Design
ist eine Messung mit Stromquellen gut möglich.

\subsection{RC-Oszillator}
Mit einem Timer-Baustein, wie z.B. LM555 wird ein Oszillator aufgebaut,
dessen Oszillatorfrequenz gemessen wird. Für den LM555 beträgt die Frequenz

\begin{equation*}
	f = \frac{1}{0.693 \cdot C_1 \cdot (R_1 + 2 R_2) }
\end{equation*}

\subsection{AC-Widerstand / Phasenverschiebung}
Die Phasenverschiebung wird gemessen, indem mit einer Sinusquelle das RC-Netzwerk,
bestehend aus der Messkapazität $C_x$ und einem bekannten Widerstand $R$,
angeregt wird und die Phasenverschiebung mittels eines Lock-In Verstärkers
gemessen wird.

\begin{align*}
	& \text{Eingangssignal} & f_{\In}(\varphi,t) &= \sin(\omega_0 t + \varphi) \\
	& \text{Sync-Signal} & f_{\text{sync}}(t) &= \sin(\omega_0 t) \\
	& \text{Produkt} & f_{\text{mod}}(\varphi,t) &= \sin(\omega_0 t + \varphi) \cdot \sin(\omega_0 t) \\
	& & &= \frac{\cos(\varphi)}{2} - \frac{\cos(\varphi + 2 t \omega_0)}{2}
\end{align*}

Der Mittelwert von $f_{\text{mod}}$ über eine Periode beträgt also
\begin{equation*}
	\bar{f}_{\text{mod}} = \frac{\cos(\varphi)}{2}
\end{equation*}
womit sich die Phase des Eingangssignals $f_{\In}$ bestimmen lässt. \\

Das gleiche Verfahren funktioniert auch mit einem Synchrongleichrichter. 
Dabei wird das Eingangssignal mit einem zum Sinus phasengleichen Rechtecksignal
multipliziert:
\begin{equation*}
	f_{\text{mod}}(\varphi,t)) = \text{sgn}\left(\sin(\omega_0 t)\right) \cdot \sin(\omega_0 t + \varphi)
\end{equation*}
Damit ergibt sich über eine Periode integriert
\begin{equation*}
	\bar{f}_{\text{mod}} = \frac{2 \cos(\varphi)}{\pi}
\end{equation*}

Damit werden Frequenzen geradzahliger Vielfacher ($2,4,...$) vom Integrator
ausgelöscht, während ungeradzahlige Vielfache um den Faktor $1/n$
abgeschwächt werden.

\subsection{Ladungsverstärker}
Eine Kapazität $C_x$ wird mit einer bekannten Spannung geladen. Die resultierende
Ladung beträgt
\begin{equation*}
	Q_x = C_x \cdot V_{\Ref}
\end{equation*}
Die Kapazität wird nun auf eine bekannte Kapazität $C_{\Ref}$ entladen. 
Es resultiert die Ausgangsspannung
\begin{equation*}
	V_{\Out} = - \frac{Q_x}{C_{\Ref}} = \frac{C_x}{C_{\Ref} V_{\Ref}}
\end{equation*}
\section{Aktive Filter}

\subsection{Filter 1. Ordnung}

\subsubsection{Integrator}
\begin{multicols}{2}
	\begin{center}
		\includegraphics[width=6cm]{images/filter_1_integrator.jpg}
	\end{center}
	\begin{align*}
		T(s) &= Y_{\In} \Zop = - \frac{1}{s C_f R_1} \\
		V_{\Out} &= -\frac{1}{R_1 C_f} \int V_{\In} dt + V_0
	\end{align*}
\end{multicols}

\subsubsection{Tiefpass}
\begin{multicols}{2}
	\begin{center}
		\includegraphics[width=6cm]{images/filter_1_tiefpass.jpg}
	\end{center}
	\begin{align*}
		T(s) &= -\frac{R_f}{R_1} \cdot \frac{1}{1 + s C_f R_f} \\
		fc &= \frac{1}{2 \pi R_f C_f}
	\end{align*}
\end{multicols}

\newpage
\subsubsection{Differenzierer}
\begin{multicols}{2}


	Achtung: dieser Differenzierer schwingt meistens.
	\begin{center}
		\includegraphics[width=6cm]{images/filter_1_diff.jpg}
	\end{center}
	\begin{align*}
		T(s) = -s C_1 R_f
	\end{align*}
\vfill	
\columnbreak
	Bandpass mit differenzierendem Frequenzbereich
	\begin{center}
		\includegraphics[width=0.7\linewidth]{images/filter_1_diff2.jpg}
	\end{center}
	\begin{align*}
		T(s) = - \frac{s C_1 R_f}{1 + s C_1 R_1} \cdot \frac{1}{1 + s C_f R_f}
	\end{align*}
	Differenziert zwischen $\omega_1 = \frac{1}{R_1 C_1}$ und 
	$\omega_2 = \frac{1}{R_f C_f}$

\end{multicols}

\subsection{Filter höherer Ordnung}

\begin{multicols}{2}
	\paragraph{Tiefpass 2. Ordnung}
	\begin{equation*}
		T_{TP}(s) = \frac{A \omega_0^2}{s^2 + \frac{\omega_0}{Q} s + \omega_0^2}
	\end{equation*}
	
	\paragraph{Bandpass 2. Ordnung}
	\begin{equation*}
		T_{BP}(s) = \frac{A \frac{\omega_0}{Q} s}{s^2 + \frac{\omega_0}{Q} s + \omega_0^2}
	\end{equation*}
\end{multicols}


\subsubsection{Sallen-Key Filter}
\paragraph{Sallen-Key Tiefpass}
\begin{center}
	\includegraphics[width=8cm]{images/filter_sk_tp.jpg}
\end{center}
\begin{equation*}
	T(s) = \frac{A}{1 + \left(R_3 C_4 + R_1 C_4 + R_1 C_2 (1-A)\right) \cdot s + R_1 R_3 C_2 C_4 \cdot s^2}
\end{equation*}

\paragraph{Sallen-Key Bandpass}
\begin{center}
	\includegraphics[width=8cm]{images/filter_sk_bp.jpg}
\end{center}
\begin{equation*}
	T(s) = \frac{\frac{R_2 R_5}{R_1 + R_2} C_4 A_M \cdot s}{1 + \frac{R_1 R_5 C_4 (1-A_M) + R_1 R_2 (C_3 + C_4) + R_2 R_5 C_4}{R_1 + R_2} \cdot s + \frac{R_1 R_2}{R_1 + R_2} R_5 C_3 C_4 \cdot s^2}
\end{equation*}
mit $A_M = 1+\frac{R_7}{R_6}$

\subsubsection{Multiple Feedback Filter}
\begin{center}
	\includegraphics[width=8cm]{images/filter_mfb.jpg}
\end{center}
\begin{align*}
	T(s) &= \frac{G_0}{1 + C_s \left(R_2 + R_3 + R_3 \frac{R_2}{R_1}\right) \cdot s + C_1 C_2 R_2 R_3 \cdot s^2} \qquad \text{mit} \quad G_0 = -\frac{R_2}{R_1} \\
	Q &= \frac{\sqrt{C_1 C_2 R_2 R_3}}{C_2 \left(R_2 R_3 R_3 \frac{R_2}{R_1}\right)}
\end{align*}
Die Güte wird v.a. mit $C_2$ und $R_1$ eingestellt.


\subsection{Biquads}
Beispiel Bandpass:
\begin{center}
	\includegraphics[width=8cm]{images/filter_biquad.jpg} 
	\hspace{2cm}
	\includegraphics[width=8cm]{images/filter_biquad_block.jpg} 
\end{center}
\begin{align*}
	T(s) &= \frac{Y_{\In} Z_{\Int 1} Y_i Z_i Y_{r 2} Z_{\Int 2}}{1 - (Y_{\Fb 1}+Y_{\Fb 2}) Z_{\Int 1} Y_i Z_i Y_{r 2} Z_{\Int 2}}
	&= \frac{\frac{C_{bp}}{C_{i1} R_{i2} C_{i2}} \cdot s}{s^2 + \frac{C_{fb}}{C_{i1} R_{i2} C_{i2}} \cdot s + \frac{1}{R_{fb} C_{i1} R_{i2} C_{i2}}}
\end{align*}
und damit
\begin{equation*}
	A = -\frac{C_{bp}}{C_{fb}} \qquad Q = \sqrt{\frac{R_{i2}}{R_{fb}}} \cdot \frac{\sqrt{C_{i1} C_{i2}}}{C_{fb}} \qquad f_0 = \frac{1}{2 \pi \sqrt{C_{i1} C_{i2} R_{fb} R_{i2}}}
\end{equation*}

%TODO: Tiefpass, Abgeänderte Version, Tow Thomas, Ackerberg-Mossberg
\section{Switched Capacitor Schaltungstechnik}

\subsection{Grundlagen}

\begin{multicols}{2}
	\paragraph{Grundschaltung}~\\
		\adjustbox{scale=1}{% Create the switch symbol
\makeatletter
\pgfcircdeclarebipole{}{\ctikzvalof{bipoles/interr/height 2}}{spst}{\ctikzvalof{bipoles/interr/height}}{\ctikzvalof{bipoles/interr/width}}{

    \pgfsetlinewidth{\pgfkeysvalueof{/tikz/circuitikz/bipoles/thickness}\pgfstartlinewidth}

    \pgfpathmoveto{\pgfpoint{\pgf@circ@res@left}{0pt}}
    \pgfpathlineto{\pgfpoint{.6\pgf@circ@res@right}{\pgf@circ@res@up}}
    \pgfusepath{draw}   
}
% make the shape accessible with nice syntax
\def\pgf@circ@spst@path#1{\pgf@circ@bipole@path{spst}{#1}}
\tikzset{switch/.style = {\circuitikzbasekey, /tikz/to path=\pgf@circ@spst@path, l=#1}}
\tikzset{spst/.style = {switch = #1}}
\makeatother

\begin{circuitikz}[scale=0.5]

	\draw
		(0,-2) node[sground]{} to[american voltage source] (0,-0.5)
		(0,-0.5)to (0,0)
		(0,0) 	to[short,i=$i_{in}$] (1,0)
		(1,0)	to[switch=$S_0$] (2,0)
		(2,0)	to (4,0)
		(3,0)	to[short,i=$i_c$](3,-0.5)
		(3,-0.5)to[C=$C$] (3,-2) node[sground]{}
		(4,0)	to[switch=$S_1$] (5,0)
		(5,0)	to (6,0)
		(6,0)	to[short,i=$i_{out}$] (6,-2) node[sground]{}
		
		;
		
	

\end{circuitikz}} 
	
	\vfill
	\columnbreak
	
	\paragraph{Timing}~\\
	\begin{center}
		\adjustbox{scale=1.2}{\begin{tikztimingtable}
	PH0	& l 2H  2L  2H  2L  2H L \\
	PH1	& l 2L lhhl 2L lhhl 2L lh\\
	\extracode
	\node[anchor=south] at (1.5,1.2) {o};
	\node[anchor=south] at (3.5,1.2) {e};
	\node[anchor=south] at (5.5,1.2) {o};
	\node[anchor=south] at (7.5,1.2) {e};
	\node[anchor=south] at (9.5,1.2) {o};
	\endextracode
\end{tikztimingtable}
} \\
		Wichtig: non-overlapping \\
		e: even phase \\
		o: odd phase
	\end{center}

\end{multicols}
	
\begin{tabular}{lllll}
	Phase PH0 & $S_0$ geschlossen & $S_1$ offen & $I_c = I_{\In} = \frac{V_{\In}}{R_s} e^{-\frac{t}{R_s C}}$ & \\
	Phase PH1 & $S_0$ offen & $S_1$ geschlossen & $I_c = -I_{\Out} = \frac{V_{\In}}{R_s} e^{-\frac{t^*}{R_s C}}$ & mit $t^* = t - \frac{T}{2} \quad T=\frac{1}{\Fclk}$ \\
	Durchschn. Strom & & & $I_d = \frac{\Delta Q}{T} = V_{\In} \frac{C}{T}$ \\
	Äquiv. Widerstand & & & $R_{eq} = \frac{T}{C}$ & weil $I_{d,\text{res}} = \frac{V_{\In}}{R}$ \\
\end{tabular}
	 
\subsection{z-Transformation}
Für sehr hohe Taktfrequenzen können sC-Schaltungen mit der Laplace-Transformation beschrieben werden.
Sonst muss die z-Transformation verwendet werden.

\begin{tabularx}{\linewidth}{p{0.25\linewidth}XX}
	\textbf{Bezeichnung} & \textbf{Schaltung} & \textbf{Admittanz} \\ \hline
	Kapazität & \begin{circuitikz}[scale=0.7,transform shape]
	\draw
		(0,0) node[anchor=east]{$V_{\In}$} to[C=$C$]  (6,0) node[anchor=west]{$V_{\Out}$}
		;	

\end{circuitikz} & $Y_{c}(z) = C \cdot (1-z^{-1})$ \\
	Geschaltetes C & \input{tikz/block_sc1} & $Y_{sc}(z) = C$ \\
	Geschaltetes C mit Inversion & \begin{circuitikz}[scale=0.7,transform shape]
	\draw
		(0,0) node[anchor=east]{$V_{\In}$} to[switch={PH0}]  (2,0)
		(2,0) to[switch={PH1},mirror] (2,-1) node[sground,scale=0.7]{}
		(2,0) to[C=$C$] (4,0)
		(4,0) to[switch={PH0}] (4,-1) node[sground,scale=0.7]{}
		(4,0) to[switch={PH1}] (6,0) node[anchor=west]{$V_{\Out}$}
		;	

\end{circuitikz} & $Y_{sc}^{eo}(z) = -C \cdot z^{-1/2}$ \newline $Y_{sc}^{ee}(z) = -C \cdot z^{-1}$ \\
	\hline
\end{tabularx}

\begin{tabularx}{\linewidth}{p{0.25\linewidth}XX}
	\textbf{Bezeichnung} & \textbf{Schaltung} & \textbf{Impedanz} \\ \hline
	Opamp als Integrator & \input{tikz/block_opi} & $\Zop(z) = - \frac{1}{C_f} \frac{1}{1-z^{-1}}$ \\
	Opamp als Tiefpass & \begin{circuitikz}[scale=0.7,transform shape]
	\draw

		(0,0) node[op amp] (opamp) {}

		(-2.5,0.49) node[anchor=east] {$V_{\In}$} to (opamp.-)
		(-2,0.49) to[short,*-] (-2,1.5)
		(-2,1.5) to[C=$C_f$] (2,1.5)
		(-2,1.5) to[short,*-] (-2,3)
		(2,1.5) to[short,*-] (2,3)
				
		
		%(-1.5,3) to[R=$R_f$] (1.5,3)
			(-2,3) to[switch={\small PH 0}]  (-1,3)
			(-1,3) to[switch={\small PH 1},mirror] (-1,2) node[sground,scale=0.7]{}
			(-1,3) to[C=$C_r$] (1,3)
			(1,3) to[switch={\small PH 1}] (1,2) node[sground,scale=0.7]{}
			(1,3) to[switch={\small PH 0}] (2,3)
		
		(2,1.5) to[short,-*] (2,0)
		(opamp.out) to (2.5,0) node[anchor=west] {$V_{\Out}$}
		(opamp.+) -| (-2,-1) node[sground,scale=0.7] {}
				
	;	

\end{circuitikz}& $\Zop(z) = - \frac{\frac{1}{C_f + C_r}}{1-\frac{C_f}{C_f+C_r}z^{-1}}$ \\
	\hline

\end{tabularx}
\section{Spezielle Operationsverstärker}

\subsection{Unkompensierte Opamps}

\begin{minipage}{9cm}
	\includegraphics[width=7cm]{images/op_uncomp.jpg}
\end{minipage}
\begin{minipage}{-9cm+\linewidth}
	Während normale Opamps bis zu Gain $\pm 1$ stabil sind, benötigen
	unkompensierte Opamps einen minimalen Gain $G_{min}$ (z.B. $10$) um
	stabil zu sein. \\
	
	Die Pole werden durch die fehlende Kompensation höher liegen, womit
	ein höheres Gain-Bandwith-Product erreicht wird (z.B. 88 MHz statt 17 MHz)
\end{minipage}


\subsection{Current Feedback Amplifier (CFA)}
VFA und CFA werden gleich beschaltet, mit einem CFA sind jedoch höhere
Bandbreiten bei höherer Verstärkung möglich. CFA werden daher für schnelle
Treiber eingesetzt, und oft in Bipolar-Technologie aufgebaut. Der Nachteil
des CFA ist die grössere Offset-Spannung. \\

\begin{tabularx}{\linewidth}{p{3cm}XX}
	&\textbf{Voltage Feedback Opamp (VFA)} & \textbf{Current Feedback Opamp (CFA)} \\
	\hline
	& \includegraphics[width=3cm]{images/op_vfa.jpg} & \includegraphics[width=3cm]{images/op_cfa.jpg} \\
	Eigenschaften: & Hochohmige Eingänge \newline hohe Spannungsverstärkung $A$ & Buffer von $V_p$ nach $V_n$ \newline $V_p$: hochohmiger Eingang, $V_n$: niederohmiger Ausgang \\
	Invertierender Verstärker & $b = \frac{R_1}{R_1 + R_2}$ \newline $\frac{V_o}{V_i} = \frac{1}{b} \frac{1}{1+\frac{1}{a b}}$ \newline Error term $\frac{1}{ab}$ ist abhängig vom Gain. & $b = \frac{R_1}{R_1 + R_2}$ \newline $\frac{V_o}{V_i} = \frac{1}{b} \frac{1}{1 + \frac{R_2}{Z_t}}$ \newline Error term $\frac{R_2}{Z_t}$ ist unabhängig von $R_1$ und damit vom Gain $1/b$. \\
	& $f_c = \frac{g_m}{2 \pi C_c} \frac{R_1}{R_1 + R_2}$ \newline grösserer Gain $\Rightarrow$ kleinere Bandbreite & $\frac{V_o}{V_i} \simeq \frac{R_1 + R_2}{R_1} \frac{1}{1 + j 2 \pi f R_2 C_c}$ \newline Bandbreite nur von $C_c$ und $R_2$ abhängig. \\
	\hline
\end{tabularx}


\subsection{Diamond Transistor}
Der Operational Transconductance Amplifier (OTA), Diamond Transistor oder
Voltage controlled current source, kann wie ein (fast) idealer Transistor
betrachtet werden. 

\begin{center}
	\includegraphics[width=12cm]{images/op_diamond}
\end{center}

\begin{tabular}{lllll}
	\textbf{NPN Transistor} & $V_{BE} \approx 0.7V$ & 
		$g_m$ nichtlinear $(g_m \sim I_C)$ & $I_C > 0$ & $I_E \approx-I_C$ \\
	\textbf{Diamond Transistor} & $V_{BE} = 0V$ & $I_C = g_m V_{BE}$ &
		$I_C$ positiv und negativ & $I_E = I_C$ \\
\end{tabular}

ToDo: gmC Filter
\section{Sigma-Delta Wandler}

\begin{multicols}{2}
	\paragraph{Zeitkontinuierliches Modell} ~\\
	\includegraphics[width=0.8\linewidth]{images/sigma_delta.jpg}
	\begin{align*}
		\text{Signal UTF:} \quad & H_s(s) = \frac{1}{1+sT} \\
		\text{Noise UTF:} \quad & H_n(s) = \frac{sT}{1+sT}
	\end{align*}
	\vfill\columnbreak
	\paragraph{Zeitdiskretes Modell} ~\\
	\includegraphics[width=0.8\linewidth]{images/sigma_delta_diskret.jpg}
	\begin{align*}
		\text{Signal UTF:} \quad & H_s(z) = z^{-1} \\
		\text{Noise UTF:} \quad & H_n(z) = 1-z^{-1}
	\end{align*}
\end{multicols}

\subsection{Pattern Noise}
DC-Eingangsspannungen führen zu repetitiven Sequenzen am Modulator-Ausgang, so genanntem Pattern Noise.
Ist $x$ sehr klein, entstehen repetitive Sequenzen tiefer Frequenz.
\begin{center}
	\begin{tabular}{ll}
		\textbf{Eingangsspannung} & \textbf{Pattern Noise Periode} \\ \hline
		$V_{\In} = 0$ & $\frac{1}{2} \Fclk$\\
		$V_{\In} = \pm \frac{1}{2} V_{\Ref}$ & $\frac{1}{4} \Fclk$ \\
		$V_{\In} = \pm \frac{1}{8} V_{\Ref}$ & $\frac{1}{16} \Fclk$ \\
		$V_{\In} = \pm 0.1 V_{\Ref}$ & $\frac{1}{20} \Fclk$ \\
		$V_{\In} = x \cdot V_{\Ref}$ & $\frac{x}{2} \Fclk$\\
		\hline
	\end{tabular}
\end{center}


\subsection{Signal-Rausch-Abstand}

Für einen $n$-bit Modulator bei einer Signalfrequenz von $f_0$ kann das Rauschen wie folgt berechnet werden:
\begin{flalign*}
	&\text{Signal-to-Noise Ratio} && \text{SNR} && \approx -3.4 + 6 n + 9 \log_2\left(OSR\right) \\
	&\text{Oversampling Ratio} && \text{OSR} && = \frac{f_s / 2}{f_0} \\
	&\text{Effektivwert $n_0$ der Rausch-Spannung für Ordnung $n=1$} && n_0 && \approx \frac{q}{\sqrt{12}} \frac{\pi}{\sqrt{3}} \left(\frac{2 f_0}{f_s}\right)^{3/2} \\
	&\text{Effektivwert $n_0$ der Rausch-Spannung für Ordnung $n=2$} && n_0 && \approx \frac{q}{\sqrt{12}} \frac{\pi^2}{\sqrt{5}} \left(\frac{2 f_0}{f_s}\right)^{5/2} \\
	&\text{Effektivwert $n_0$ der Rausch-Spannung für Ordnung $n=3$} && n_0 && \approx \frac{q}{\sqrt{12}} \frac{\pi^3}{\sqrt{7}} \left(\frac{2 f_0}{f_s}\right)^{7/2} \\
\end{flalign*}
%\input{sections/sensor_systeme}
\section{Analyse von Schaltungen}
Die Übertragungsfunktion von Opamp Schaltungen berechnet sich als 
Summe aller Eingangs-Admittanzen $Y_n$ multipliziert mit der Opamp-Funktion
$\Zop$. 
\begin{multicols}{2}
	\begin{center}
		\begin{tikzpicture}[scale=0.7]

% Nodes
\node (vin) at (-2,0) {$V_{\In}$};

\node[rectangle,draw] (y1) at (0,1) {$Y_1$};
\node at (0,0) {$\vdots$};
\node[rectangle,draw] (yn) at (0,-1) {$Y_n$};

\node[circle,draw] (sum) at (2,0) {$+$};
\node[rectangle,draw] (zop) at (4,0) {$\Zop$};
\node (vout) at (6,0) {$V_{\Out}$};

% Connections
\draw[thick,->] (vin) -- (y1);
\draw[thick,->] (vin) -- (yn);

\draw[thick,->] (y1) -- (sum);
\draw[thick,->] (yn) -- (sum);

\draw[thick,->] (sum) -- (zop);
\draw[thick,->] (zop) -- (vout);

\end{tikzpicture}
	\end{center}
	\vfill
	\columnbreak
	\hspace{2cm}
	\begin{equation*}
		T(z) = \sum\limits_{n} Y_{n} \Zop
	\end{equation*}
\end{multicols}

\subsection{Spannung zu Strom}

\begin{tabularx}{\linewidth}{p{0.25\linewidth}XX}
	\textbf{Bezeichnung} & \textbf{Schaltung} & \textbf{Admittanz} \\ \hline
	Widerstand & \begin{circuitikz}[scale=0.7,transform shape]
	\draw
		(0,0) node[anchor=east]{$V_{\In}$} to[R=$R$]  (6,0) node[anchor=west]{$V_{\Out}$}
		;	

\end{circuitikz} & $Y_{r}(s) = \frac{1}{R}$ \\
	Kapazität & \begin{circuitikz}[scale=0.7,transform shape]
	\draw
		(0,0) node[anchor=east]{$V_{\In}$} to[C=$C$]  (6,0) node[anchor=west]{$V_{\Out}$}
		;	

\end{circuitikz} & $Y_{c}(s) = sC$ \\
	Geschaltetes C & \input{tikz/block_sc1} & $Y_{sc}(s) = \frac{C}{T}$ \\
	Geschaltetes C mit Inversion & \begin{circuitikz}[scale=0.7,transform shape]
	\draw
		(0,0) node[anchor=east]{$V_{\In}$} to[switch={PH0}]  (2,0)
		(2,0) to[switch={PH1},mirror] (2,-1) node[sground,scale=0.7]{}
		(2,0) to[C=$C$] (4,0)
		(4,0) to[switch={PH0}] (4,-1) node[sground,scale=0.7]{}
		(4,0) to[switch={PH1}] (6,0) node[anchor=west]{$V_{\Out}$}
		;	

\end{circuitikz} & $Y_{sc}(s) = -\frac{C}{T}$ \\
	\hline
\end{tabularx}

\subsection{Strom zu Spannung}

\begin{tabularx}{\linewidth}{p{0.25\linewidth}XX}
	\textbf{Bezeichnung} & \textbf{Schaltung} & \textbf{Impedanz} \\ \hline
	Opamp als Verstärker & \begin{circuitikz}[scale=0.7,transform shape]
	\draw

		(0,0) node[op amp] (opamp) {}

		(-2,0.49) node[anchor=east] {$V_{\In}$} to (opamp.-)
		(-1.5,0.49) to[short,*-] (-1.5,1.5)
		(-1.5,1.5) to[R=$R_f$] (1.5,1.5)
		(1.5,1.5) to[short,-*] (1.5,0)
		(opamp.out) to (2,0) node[anchor=west] {$V_{\Out}$}
		(opamp.+) -| (-1.5,-1) node[sground,scale=0.7] {}
				
	;	

\end{circuitikz} & $\Zop = -R_f$ \\
	Opamp als Integrator & \input{tikz/block_opi} & $\Zop = - \frac{1}{s C_f}$ \\
	Opamp als Tiefpass & \input{tikz/block_opt} & $\Zop = - \frac{R_f}{1+s C_f R_f}$  \\
	\hline
\end{tabularx}

\subsection{Rechenregeln Blockdiagramme}

\begin{tabularx}{\linewidth}{p{0.25\linewidth}XX}
	\textbf{Bezeichnung} & \textbf{Schaltung} & \textbf{Berechnung} \\ \hline
	Serieschaltung & \begin{circuitikz}[scale=0.7,transform shape]
	\node[anchor=east] (xs) at (0,0) {$X(s)$};
	\node[draw,rectangle] (t1) at (2,0) {$T_1(s)$};
	\node[draw,rectangle] (t2) at (4,0) {$T_2(s)$};
	\node[anchor=west] (ys) at (6,0) {$Y(s)$};		

	\draw[thick,->] (xs) -- (t1);
	\draw[thick,->] (t1) -- (t2);
	\draw[thick,->] (t2) -- (ys);

	\node at (0,-1) {}; \node at (0,1) {};
\end{circuitikz} & $T(s) = T_1(s) \cdot T_2(s)$ \\
	Parallelschaltung & \begin{circuitikz}[scale=0.7,transform shape]
	\node[anchor=east] (xs) at (0,1) {$X(s)$};
	\node[draw,rectangle] (t1) at (2,1) {$T_1(s)$};
	\node[draw,rectangle] (t2) at (2,-1) {$T_2(s)$};
	\node[anchor=west] (ys) at (6,1) {$Y(s)$};	
	\node[draw,circle,minimum size=0mm,inner sep=2pt] (sum) at (4,1) {$+$};
	\node[circle,fill,inner sep=0pt, minimum size=2mm] (p) at (0.75,1) {};	

	\draw[thick,->] (xs) -- (t1);
	\draw[thick,->] (p) |- (t2.west);
	\draw[thick,->] (t1) -- (sum.west);
	\draw[thick,->] (t2.east) -| (sum.south);
	\draw[thick,->] (sum.east) -- (ys);

\end{circuitikz} & $T(s) = T_1(s) + T_2(s)$ \\
	Rückkopplung & \begin{circuitikz}[scale=0.7,transform shape]
	\node[anchor=east] (xs) at (0,1) {$X(s)$};
	\node[draw,circle,minimum size=0mm,inner sep=2pt] (sum) at (2,1) {$+$};
	\node[draw,rectangle] (t1) at (4,1) {$T_1(s)$};
	\node[draw,rectangle] (t2) at (4,-1) {$T_2(s)$};
	\node[anchor=west] (ys) at (6,1) {$Y(s)$};	

	\node[circle,fill,inner sep=0pt, minimum size=2mm] (p) at (5,1) {};	

	\draw[thick,->] (xs) -- (sum);
	\draw[thick,->] (sum) -- (t1);
	\draw[thick,->] (t1) -- (ys);
	\draw[thick,->] (p) |- (t2);
	\draw[thick,->] (t2) -| (sum);

\end{circuitikz} & $T(s) = \frac{T_1(s)}{1 - T_1(s) T_2(s)}$ \\
	Mason & 
		$T(s) = \frac{\sum\limits_{i=1}^N T_i \Delta_i}{\Delta}$ \newline
		~\newline
		$\Delta= 1 - \sum L_i +\sum L_i L_j -+ \ldots$ \newline
		$\Delta_i$ wie $\Delta$, jedoch ohne Loops die den Pfad $i$ berühren
		& 
		$N$: Anzahl Vorwärtspfade \newline 
		$T_i$: Vorwärtspfad $i$ \newline
		$\Delta$: Determinante \newline 
		$L_i$: Geschlossene Loops \newline 
		$L_i L_j$: Zwei sich nicht berührende Loops \\
	\hline
\end{tabularx}

\appendix
\section{Idiotenseite}
\subsection{SI-Präfixe}
\begin{tabular}{lllll}
	Name	& Potenz& \qquad	& Name & Potenz \\ \hline
	m 		& -3 	&	& k	& 3	\\
	$\mu$	& -6	&	& M	& 6	\\
	n 		& -9 	&	& G	& 9	\\
	p 		& -12 	&	&		&	\\
	f 		& -15	&	&		&	\\ \hline
\end{tabular}

\end{document}
