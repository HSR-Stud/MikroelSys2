\section{Systembetrachtung elektronischer Schaltungen}
\subsection{Blockdiagramme}
\subsubsection{Verknüpfung}
\begin{tabular}{ll}
Serieschaltung & $G_{total}=G_1G_2$\\
Parallelschaltung & $G_{total}=G_1+G_2$\\
Rückkopplung & $G_{total}=\frac{G_V}{(1+G_VG_R)}$\\
\end{tabular}
\subsubsection{Formel von Mason}
$T=\frac{\Sigma_iP_i\Delta_i}{\Delta}$, mit
\begin{tabular}{ll}
$P_i$ & Vorwärtspfade durch das Diagramm \\
$\Delta$ & Graphendeterminante \\
\end{tabular}\\
\textbf{Graphendeterminante}
\begin{itemize}
\item 1
\item minus Kreisverstärkung aller Kreise $L_i$
\item plus das Produkt der Kreisverstärkungen von jeweils zwei Kreisen, die keine Knoten gemeinsam haben
\item minus das Produkt der Kreisverstärkungen von jeweils drei Kreisen, die keine Knoten gemeinsam haben
\item plus das Produkt der Kreisverstärkungen von jeweils vier Kreisen, die keine Knoten gemeinsam haben
\item etc.
\end{itemize}
\subsection{Übertragungsfunktion}
\begin{tabular}{ll}
UTF & $G(s)=\frac{V_{out}(s)}{V_{in}(s)}$\\
Amplitute & $|G|=\sqrt{Re(G)^2+Im(G)^2}$\\
Phase & $\varphi\{G\}=arctan\frac{Im(G)}{Re(G)}$\\
\end{tabular}\\
%TODO hier könnte man etwas mehr schrieben. Wenn man Lust hätte
\subsection{Bodediagramme}
\subsubsection{Vorteile der Bode-Darstellung}
Amplitudenabfall linear mit 20dB/Dekade\\
Grenzfrequenz beim Knickpunkt sichtbar\\
Multipkikation zweier UTF: Amplituden- und Phasengänge werden addiert\\
Division zweier UTF: Amplituden- und Phasengänge werden subtrahiert\\
\subsubsection{Elementare Übertragungsglieder}
Siehe Tabelle Seiten 2-19 und 2-20
\subsubsection{Grafische Eigenschaften und Approximationen}
Seite 2-21
