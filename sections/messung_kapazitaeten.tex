\section{Messung von Kapazitäten}

\subsection{Stromquelle}
Prinzip: Kapazität wird mit einem Strom $I_0$ während einer fixen Zeit 
$t_0$ aufgeladen. Die Kapazität berechnet sich aus
\begin{equation*}
	V_{\Out} = C_x \cdot I_0 \cdot t_0
\end{equation*}
Diese Art der Messung benötigt eine präzise Stromquelle oder eine
Referenzkapazität. Weiter werden sehr kleine Ströme benötigt, was
diese Methode für PCB-Design unpraktikabel macht. Im IC-Design
ist eine Messung mit Stromquellen gut möglich.

\subsection{RC-Oszillator}
Mit einem Timer-Baustein, wie z.B. LM555 wird ein Oszillator aufgebaut,
dessen Oszillatorfrequenz gemessen wird. Für den LM555 beträgt die Frequenz

\begin{equation*}
	f = \frac{1}{0.693 \cdot C_1 \cdot (R_1 + 2 R_2) }
\end{equation*}

\subsection{AC-Widerstand / Phasenverschiebung}
Die Phasenverschiebung wird gemessen, indem mit einer Sinusquelle das RC-Netzwerk,
bestehend aus der Messkapazität $C_x$ und einem bekannten Widerstand $R$,
angeregt wird und die Phasenverschiebung mittels eines Lock-In Verstärkers
gemessen wird.

\begin{align*}
	& \text{Eingangssignal} & f_{\In}(\varphi,t) &= \sin(\omega_0 t + \varphi) \\
	& \text{Sync-Signal} & f_{\text{sync}}(t) &= \sin(\omega_0 t) \\
	& \text{Produkt} & f_{\text{mod}}(\varphi,t) &= \sin(\omega_0 t + \varphi) \cdot \sin(\omega_0 t) \\
	& & &= \frac{\cos(\varphi)}{2} - \frac{\cos(\varphi + 2 t \omega_0)}{2}
\end{align*}

Der Mittelwert von $f_{\text{mod}}$ über eine Periode beträgt also
\begin{equation*}
	\bar{f}_{\text{mod}} = \frac{\cos(\varphi)}{2}
\end{equation*}
womit sich die Phase des Eingangssignals $f_{\In}$ bestimmen lässt. \\

Das gleiche Verfahren funktioniert auch mit einem Synchrongleichrichter. 
Dabei wird das Eingangssignal mit einem zum Sinus phasengleichen Rechtecksignal
multipliziert:
\begin{equation*}
	f_{\text{mod}}(\varphi,t)) = \text{sgn}\left(\sin(\omega_0 t)\right) \cdot \sin(\omega_0 t + \varphi)
\end{equation*}
Damit ergibt sich über eine Periode integriert
\begin{equation*}
	\bar{f}_{\text{mod}} = \frac{2 \cos(\varphi)}{\pi}
\end{equation*}

Damit werden Frequenzen geradzahliger Vielfacher ($2,4,...$) vom Integrator
ausgelöscht, während ungeradzahlige Vielfache um den Faktor $1/n$
abgeschwächt werden.

\subsection{Ladungsverstärker}
Eine Kapazität $C_x$ wird mit einer bekannten Spannung geladen. Die resultierende
Ladung beträgt
\begin{equation*}
	Q_x = C_x \cdot V_{\Ref}
\end{equation*}
Die Kapazität wird nun auf eine bekannte Kapazität $C_{\Ref}$ entladen. 
Es resultiert die Ausgangsspannung
\begin{equation*}
	V_{\Out} = - \frac{Q_x}{C_{\Ref}} = \frac{C_x}{C_{\Ref} V_{\Ref}}
\end{equation*}