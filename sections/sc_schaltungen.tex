\section{Switched Capacitor Schaltungstechnik}

\subsection{Grundlagen}

\begin{multicols}{2}
	\paragraph{Grundschaltung}~\\
		\adjustbox{scale=1}{% Create the switch symbol
\makeatletter
\pgfcircdeclarebipole{}{\ctikzvalof{bipoles/interr/height 2}}{spst}{\ctikzvalof{bipoles/interr/height}}{\ctikzvalof{bipoles/interr/width}}{

    \pgfsetlinewidth{\pgfkeysvalueof{/tikz/circuitikz/bipoles/thickness}\pgfstartlinewidth}

    \pgfpathmoveto{\pgfpoint{\pgf@circ@res@left}{0pt}}
    \pgfpathlineto{\pgfpoint{.6\pgf@circ@res@right}{\pgf@circ@res@up}}
    \pgfusepath{draw}   
}
% make the shape accessible with nice syntax
\def\pgf@circ@spst@path#1{\pgf@circ@bipole@path{spst}{#1}}
\tikzset{switch/.style = {\circuitikzbasekey, /tikz/to path=\pgf@circ@spst@path, l=#1}}
\tikzset{spst/.style = {switch = #1}}
\makeatother

\begin{circuitikz}[scale=0.5]

	\draw
		(0,-2) node[sground]{} to[american voltage source] (0,-0.5)
		(0,-0.5)to (0,0)
		(0,0) 	to[short,i=$i_{in}$] (1,0)
		(1,0)	to[switch=$S_0$] (2,0)
		(2,0)	to (4,0)
		(3,0)	to[short,i=$i_c$](3,-0.5)
		(3,-0.5)to[C=$C$] (3,-2) node[sground]{}
		(4,0)	to[switch=$S_1$] (5,0)
		(5,0)	to (6,0)
		(6,0)	to[short,i=$i_{out}$] (6,-2) node[sground]{}
		
		;
		
	

\end{circuitikz}} 
	
	\vfill
	\columnbreak
	
	\paragraph{Timing}~\\
	\begin{center}
		\adjustbox{scale=1.2}{\begin{tikztimingtable}
	PH0	& l 2H  2L  2H  2L  2H L \\
	PH1	& l 2L lhhl 2L lhhl 2L lh\\
	\extracode
	\node[anchor=south] at (1.5,1.2) {o};
	\node[anchor=south] at (3.5,1.2) {e};
	\node[anchor=south] at (5.5,1.2) {o};
	\node[anchor=south] at (7.5,1.2) {e};
	\node[anchor=south] at (9.5,1.2) {o};
	\endextracode
\end{tikztimingtable}
} \\
		Wichtig: non-overlapping \\
		e: even phase \\
		o: odd phase
	\end{center}

\end{multicols}
	
\begin{tabular}{lllll}
	Phase PH0 & $S_0$ geschlossen & $S_1$ offen & $I_c = I_{\In} = \frac{V_{\In}}{R_s} e^{-\frac{t}{R_s C}}$ & \\
	Phase PH1 & $S_0$ offen & $S_1$ geschlossen & $I_c = -I_{\Out} = \frac{V_{\In}}{R_s} e^{-\frac{t^*}{R_s C}}$ & mit $t^* = t - \frac{T}{2} \quad T=\frac{1}{\Fclk}$ \\
	Durchschn. Strom & & & $I_d = \frac{\Delta Q}{T} = V_{\In} \frac{C}{T}$ \\
	Äquiv. Widerstand & & & $R_{eq} = \frac{T}{C}$ & weil $I_{d,\text{res}} = \frac{V_{\In}}{R}$ \\
\end{tabular}
	 
\subsection{z-Transformation}
Für sehr hohe Taktfrequenzen können sC-Schaltungen mit der Laplace-Transformation beschrieben werden.
Sonst muss die z-Transformation verwendet werden.

\begin{tabularx}{\linewidth}{p{0.25\linewidth}XX}
	\textbf{Bezeichnung} & \textbf{Schaltung} & \textbf{Admittanz} \\ \hline
	Kapazität & \begin{circuitikz}[scale=0.7,transform shape]
	\draw
		(0,0) node[anchor=east]{$V_{\In}$} to[C=$C$]  (6,0) node[anchor=west]{$V_{\Out}$}
		;	

\end{circuitikz} & $Y_{c}(z) = C \cdot (1-z^{-1})$ \\
	Geschaltetes C & \begin{circuitikz}[scale=0.7,transform shape]
	\draw
		(0,0) node[anchor=east]{$V_{\In}$} to[switch={PH0}]  (2,0)
		(2,0) to[switch={PH1},mirror] (2,-1) node[sground,scale=0.7]{}
		(2,0) to[C=$C$] (4,0)
		(4,0) to[switch={PH1}] (4,-1) node[sground,scale=0.7]{}
		(4,0) to[switch={PH0}] (6,0) node[anchor=west]{$V_{\Out}$}
		;	

\end{circuitikz} & $Y_{sc}(z) = C$ \\
	Geschaltetes C mit Inversion & \begin{circuitikz}[scale=0.7,transform shape]
	\draw
		(0,0) node[anchor=east]{$V_{\In}$} to[switch={PH0}]  (2,0)
		(2,0) to[switch={PH1}] (2,-1) node[sground,scale=0.7]{}
		(2,0) to[C=$C$] (4,0)
		(4,0) to[switch={PH0}] (4,-1) node[sground,scale=0.7]{}
		(4,0) to[switch={PH1}] (6,0) node[anchor=west]{$V_{\Out}$}
		;	

\end{circuitikz} & $Y_{sc}^{eo}(z) = -C \cdot z^{-1/2}$ \newline $Y_{sc}^{ee}(z) = -C \cdot z^{-1}$ \\
	\hline
\end{tabularx}

\begin{tabularx}{\linewidth}{p{0.25\linewidth}XX}
	\textbf{Bezeichnung} & \textbf{Schaltung} & \textbf{Impedanz} \\ \hline
	Opamp als Integrator & \begin{circuitikz}[scale=0.7,transform shape]
	\draw

		(0,0) node[op amp] (opamp) {}

		(-2,0.49) node[anchor=east] {$V_{\In}$} to (opamp.-)
		(-1.5,0.49) to[short,*-] (-1.5,1.5)
		(-1.5,1.5) to[C=$C_f$] (1.5,1.5)
		(1.5,1.5) to[short,-*] (1.5,0)
		(opamp.out) to (2,0) node[anchor=west] {$V_{\Out}$}
		(opamp.+) -| (-1.5,-1) node[sground,scale=0.7] {}
				
	;	

\end{circuitikz} & $\Zop(z) = - \frac{1}{C_f} \frac{1}{1-z^{-1}}$ \\
	Opamp als Tiefpass & \begin{circuitikz}[scale=0.7,transform shape]
	\draw

		(0,0) node[op amp] (opamp) {}

		(-2.5,0.49) node[anchor=east] {$V_{\In}$} to (opamp.-)
		(-2,0.49) to[short,*-] (-2,1.5)
		(-2,1.5) to[C=$C_f$] (2,1.5)
		(-2,1.5) to[short,*-] (-2,3)
		(2,1.5) to[short,*-] (2,3)
				
		
		%(-1.5,3) to[R=$R_f$] (1.5,3)
			(-2,3) to[switch={\small PH 0}]  (-1,3)
			(-1,3) to[switch={\small PH 1},mirror] (-1,2) node[sground,scale=0.7]{}
			(-1,3) to[C=$C_r$] (1,3)
			(1,3) to[switch={\small PH 1}] (1,2) node[sground,scale=0.7]{}
			(1,3) to[switch={\small PH 0}] (2,3)
		
		(2,1.5) to[short,-*] (2,0)
		(opamp.out) to (2.5,0) node[anchor=west] {$V_{\Out}$}
		(opamp.+) -| (-2,-1) node[sground,scale=0.7] {}
				
	;	

\end{circuitikz}& $\Zop(z) = - \frac{\frac{1}{C_f + C_r}}{1-\frac{C_f}{C_f+C_r}z^{-1}}$ \\
	\hline

\end{tabularx}